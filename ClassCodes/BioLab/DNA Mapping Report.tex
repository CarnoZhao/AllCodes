\documentclass{article}
\usepackage{geometry}
\geometry{a4paper,scale=0.8}

\title{\bf{pBR322 Mapping by Restriction Digestion with EcoR1/HincII/PvuII and Electrophoresis}}
\author{Xun Zhao (Partner: Samantha)}
\date{March 15, 2019}

\begin{document}
    \begin{titlepage}
        \maketitle
        \setcounter{page}{0}
        \thispagestyle{empty}
    \end{titlepage}

    \renewcommand{\abstractname}{Introduction}
    \begin{abstract}
        In order to defend the attack from virus, some bacterias can produce enzymes to digest invasive DNA molecule of virus. Among these kinds of enzymes, some of them cut DNA randomly, while others called restriction enzymes only cut at specific sequence. Thus, we can use these enzymes to digest the plasmid and get DNA fragments with different length. 

        As the cut sites are fixed in a plasmid, we can reconstruct the relative positions of restriction sites of a plasmid, called mapping.

        Firstly, we digest the plasmid with all possible combinations of 3 restriction enzymes. Then, we can use the agarose gel and certain voltage to separate negatively charged DNA fragments and use loading dye to indicate their positions under UV. And their travel distances are proportional to the reciprocal of logarithm of the number of base pairs, namely, the size of molecule compared with the size of holes in gel, which allows us to draw a standard curve from a reference plasmid with known structure. Finally, we can get the length from travel distance based on the standard curve and use these length to rebuild a plasmid.
    \end{abstract}

    \begin{section}{Results}
        \begin{subsection}{Raw Result Photo}
        \end{subsection}
    \end{section}
\end{document}